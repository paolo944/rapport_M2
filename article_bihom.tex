% !TXS template
\documentclass[french]{article}
\usepackage[T1]{fontenc}
\usepackage[utf8]{inputenc}
\usepackage{lmodern}
\usepackage[a4paper]{geometry}
\usepackage{babel}
\usepackage{amsmath}
\usepackage{amsfonts}
\newtheorem{proposition}{Proposition}[section]
\newtheorem{definition}{Définition}[section]
\newtheorem{theoreme}{Théorème}[section]

\newcommand{\propositionTitre}[2]{
	\begin{proposition}[#1]
		#2
	\end{proposition}
}

\title{Ce que j'ai compris dans l'article: Gröbner bases of bihomogeneous ideals generated by polynomials of bidegree (1, 1): Algorithms and complexity de Faugère et al.}

\begin{document}
\maketitle

\section{Définitions et notations}
\begin{definition}
\item $I_{i-1}:f_i$ is spanned by $I_{i-1}$ and by the maximal minors of $\text{jac}_{\textbf{x}}(F_{i-1})$ (if $i > n_y + 1$) and $\text{jac}_{\textbf{y}}(F_{i-1})$ (if $i > n_x + 1$).
\end{definition}

\begin{definition}
Étant donné une matrice de taille $(k+1, k)$, on note minor$(\textbf{M}, j)$ le mineur obtenu en retirant la j-ième ligne de M. Considérons
$$v = (minor(jac_\textbf{x}(F), 1), -minor(jac_\textbf{x}(F), 2),
minor(jac_\textbf{x}(F), 3), -minor(jac_\textbf{x}(F), 4),)$$
\end{definition}

\begin{itemize}
	\item Soit \textbf{M} une matrice de taille $l \times c$. On appelle maximal 
	minors de \textbf{M} les déterminants des sous-matrices de taille $c \times c$
	de M.
\end{itemize}

\section{Théorèmes et propositions}
\propositionTitre{Critère F5 général}{
$$\forall j < m, \textit{si la ligne d'étiquette }(t,f_j) \textit{ dans la matrice } 
\tilde{\mathbb{M}}_{d-d_m, m-1} \textit{ a pour terme de tête, } t'$$
$$	\textit{alors la ligne } 
(t', f_m) \textit{ dans la matrice } \mathbb{M}_{d, m} \textit{ est une combinaison linéaire des précédentes.}$$
}

\propositionTitre{Critère de Frobenius}{
$$\forall j < m, \textit{si la ligne d'étiquette }(t,f_j) \textit{ dans la matrice } 
\tilde{\mathbb{M}}_{d-d_m, m} \textit{ a pour monôme de tête, } t'$$
$$	\textit{alors la ligne } 
(t', f_m) \textit{ dans la matrice } \mathbb{M}_{d, m} \textit{ est une combinaison linéaire des précédentes.}$$
}

\begin{theoreme}
Soit $i > n_x + 1 \textit{ (resp. } i > n_y + 1\textit{) }$ et soit $s$ une 
combinaison linéaire des maximal minors de $jac_\textbf{x}(F_{i-1}) 
\textit{ (resp.} jac_\textbf{y}(F_{i-1}) \textit{) }$. Alors $s \in I_{i-1}:f_i$.
\end{theoreme}



\end{document}

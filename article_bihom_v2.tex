% !TXS template
\documentclass[english]{article}
\usepackage[T1]{fontenc}
\usepackage[utf8]{inputenc}
\usepackage{lmodern}
\usepackage[a4paper]{geometry}
\usepackage{babel}
\usepackage{amsmath}
\usepackage{amsfonts}
\usepackage{mathtools}
\usepackage[backend=bibtex]{biblatex}
\usepackage{csquotes}
\addbibresource{biblio.bib}
\newtheorem{proposition}{Proposition}[section]
\newtheorem{definition}{Definition}[section]
\newtheorem{theorem}{Theorem}[section]

\newcommand{\propositionTitre}[2]{
	\begin{proposition}[#1]
		#2
	\end{proposition}
}

\author{Paul Mekhail\\ \textbf{Encadrante:} Sorina Ionica}

\title{Ce que j'ai compris jusqu'à maintenant\\Début de rapport ?}

\begin{document}
	\maketitle
	
	\section{Introduction}
		The MPC in the head paradigm is a new framework introduced in \cite{IKOS07} which is in vogue recently as we saw in NIST's second round competition for post-quantum signature schemes where 5 of the 14 candidates are based on this paradigm.
		The general idea is to use a NP-relation $\mathcal{R}(x, w)$ to obtain a ZK-protocol in which a prover $\mathcal{P}$ convinces a verifier $\mathcal{V}$ that she knows a valid witness $w$ for a given (public) value of $x$ without revealing any information about $x$.
		In \cite{HJ23}, the authors introduce a problem called subfield bilinear collision problem (SBC) which originates from the discrete logarithm problem and constructed a post-quantum signature scheme based on this problem using the MPCitH framework.
		The goal of this internship is to perform a cryptanalysis of this signature scheme.
		
		\section{Signature Scheme}
		\subsection{SBC Problem}
		The problem is the following: let $q$ be a prime power and two positive integers $k$, $n$.
		\newline
		Given two vectors $\vec{u}, \vec{v} \in (\mathbb{F}_{q^k})^n$, which are linearly independent overs $\mathbb{F_q}$, find two vectors $\vec{x}, \vec{y} \in (\mathbb{F_q})^n$ such that $$(\vec{u} \cdot \vec{x})(\vec{v} \cdot \vec{y}) = (\vec{u} \cdot \vec{y})((\vec{v} \cdot \vec{x})$$
		
		\subsection{NSBC}
		The authors also introduce a particular version of this problem called normalized SBC (NSBC) where $\vec{x}, \vec{y} \in (\mathbb{F_q})^n$ and
		$\vec{x} = (\vec{x'}, 1, 0), \vec{y} = (\vec{y}, 0, 1)$.
 		
 		\subsection{Keys and parameters estimations}
		The public key of this signature scheme is $(\vec{u},\vec{v})$ and the private key is $(\vec{x}, \vec{y})$.
		The signature and verification protocol use the MPCitH paradigm, they are described in detail in \cite{HJ23} with proof of soundness, correctness and zero-knowledge.
		The authors estimate that a good parameter for NSBC problem to be computationally secure are $q = 2, n = 130, k = 257$.
		
		\section{Cryptanalysis of the scheme}
		One of the known attacks given by the authors is through algebraic cryptanalysis.
		One can model system by taking $g$ as the following polynomial
		$$
		g(x_1,...,x_n,...,y_1,y_n) := (\vec{u} \cdot \vec{x})(\vec{v} \cdot \vec{y}) - (\vec{u} \cdot \vec{y})(\vec{v} \cdot \vec{x})
		$$
		or if we consider the NSBC problem
		$$
		g(x_1,..,x_{n-2},y_1,..,y_{n-2}) := (\sum_{1}^{n-2}u_{i}x_{i} + u_{n-1})(\sum_{1}^{n-2}v_{i}y_{i} + v_{n}) - (\sum_{1}^{n-2}u_{i}y_{i} + u_{n-1})(\sum_{1}^{n-2}v_{i}x_{i} + v_{n-1})
		$$
		
		Then, one can perform Weil descent on the polynomial $g$ because $\mathbb{F}_q^k$ is a $\mathbb{F}_q$-vector space and obtain the polynomial system
		

		\begin{align}
			g_1(x_1, \dots, x_{n-2}, y_1, \dots, y_{n-2}) &= 0 \notag \\
			&\vdots \notag \\
			g_k(x_1, \dots, x_{n-2}, y_1, \dots, y_{n-2}) &= 0 \notag
		\end{align}
		
		The most straightforward cryptanalysis of this type of systems is to perform a Gröbner basis algorithm like Faugère's F5 \cite{F02}.
		
		The authors cite \cite{FSS11} to argue their parameters choice, we will dive deeper into this article's results in a later section.

		\subsection{Gröbner basis algorithms}
		Gröbner basis algorithms are used to find solutions for polynomial systems using algebraic geometry tools. For a gentle introduction to this subject \cite{CLS}.
		
		The major "issue" of Gröbner basis algorithms is the reductions to 0, because of the matrices sizes that are huge, multiple millions for practical examples,
		reductions to 0 can cost a lot during the linear algebra process. F5 \cite{F02} was a major improvement because it provided a criterion that avoided reduction to 0 during the linear algebra part for all regular sequences.
		Bardet showed in \cite{Bardet04} that F5 criterion removed reductions to 0 for semi-regular sequences until the degree $D$ which is the degree of regularity of $I$.
		
		According to \cite{BFS15}, the complexity of computing the Gröbner basis of a system $f_1,...,f_m \in \mathbb{K}[x_1,...,x_n]$ is $O(mD\binom{n+D-1}{D}^\omega)$.
		
		In \cite{Bardet04} Theorem 3.2.10 is the following
		
		\begin{theorem}
			Let $f_1,...,f_m$ homogeneous sequence, such that $<f_1,...,f_m>$ is 0-dimensional and
			$<$ an admissible graded monomial ordering. We have,
			\begin{itemize}
				\item[-] If the sequence is semi-regular, then no reductions to 0 are performed during the F5-matrix algorithm until the degree of regularity $D - 1$.
				\item[-] If there is no reductions to 0 during the F5-matrix algorithm until the degree $D - 1$, and if the matrix of degree D is full rank and is the first matrix to have more rows than columns, then the sequence is semi-regular and it's index of regularity is $H(I) = D$.
			\end{itemize}
		\end{theorem}
		
		However, after some tests and after applying F5-matrix algorithm on polynomial systems generated by NSBC instances, we get reductions to 0, this means that the polynomial systems are not semi-regular.
		
		\subsection{Bilinear systems case}
		
			
\printbibliography

\end{document}
